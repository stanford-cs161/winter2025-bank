\documentclass{bamboo}

\Document{
	\Title{Maximum Flow}
	
	\section{Max-Flow}
	For each of the statements below select whether it is possible or not.
    
    In a residual graph $G$, we have two vertices $u$ and $v$ such that the sum of capacities on the edge from $u$ to $v$ and the edge from $v$ to $u$ is less than the initial capacity of the edge between $u$ and $v$. 
    \Choices{
        \Choice{Possible}
        \Choice*{Impossible}
    }

    There can be an instance of max-flow where some of the outgoing edges from the source $S$ and some of the incoming edges to the sink $T$ are not used to their full capacity.
    \Choices{
        \Choice*{Possible}
        \Choice{Impossible}
    }

    There exists a vertex $v$ other than the source and the sink where its incoming flow is less than its outgoing flow.
    \Choices{
        \Choice{Possible}
        \Choice*{Impossible}
    }

	Suppose we have a graph with integer capacities. Can we always find a single edge and increase its capacity by $1$ to make sure the value of max-flow increases by $1$?
    \Choices{
        \Choice{Always possible}
        \Choice*{Sometimes possible, and sometimes impossible}
        \Choice{Always impossible}
    }
    
    Suppose we have a graph with integer capacities and there is nonzero max-flow between a source and a sink vertex. We want to increase the capacity of some of the edges in order to increase the max-flow by $1$. Do we need to increase any edge's capacity by $2$ or more units?
    \Choices{
        \Choice*{Always possible while increasing edge capacities by at most $1$.}
        \Choice{Sometimes impossible to increase max-flow by $1$ with only $1$-unit edge increments.}
    }

    \section{Max-Flow Example}
    We start with a graph that has no incoming edges to the source $S$ and no outgoing edges from the sink $T$ and run a flow. The residual graph is pictured below: 
    \Tikz*{
    	\begin{scope}[every node/.style={circle, draw=Black, line width=1, fill=LightGray, inner sep=3}]
			\node (S) at (-4.5, 0) {$S$};
			\node (A) at (-1.5, 2) {$A$};
			\node (B) at (-1.5, 0) {$B$};
			\node (C) at (-1.5, -2) {$C$};
			\node (D) at (1.5, 2) {$D$};
			\node (E) at (1.5, 0) {$E$};
			\node (F) at (1.5, -2) {$F$};
			\node (T) at (4.5, 0) {$T$};
		\end{scope}
		\begin{scope}[line width=1, -stealth]
			\foreach \u/\v/\w in {A/S/4, B/A/6, E/B/4, D/E/3, A/E/1, D/A/4, T/D/7, T/E/5, E/F/2, F/C/3, E/C/2, C/S/5}
				\draw (\u) -- node[pos=0.5, fill=White] {$\w$} (\v);
			\foreach \u/\v/\w in {S/B/5, B/S/4, F/T/2, T/F/1}
				\draw (\u) to[bend left=20] node[pos=0.5, fill=White] {$\w$} (\v);
		\end{scope}

%		\foreach \u/\v/\w in {A/B/4, A/C/7, B/C/1, B/D/5, C/E/2, D/E/3, D/F/8, E/F/6}
%			\draw[line width=1] (\u) -- node[pos=0.5, fill=White] {$\w$} (\v);
    }
    % https://q.uiver.app/?q=WzAsOCxbMCwyLCJcXGJ1bGxldHtTfSJdLFsyLDAsIlxcYnVsbGV0e0F9Il0sWzIsMiwiXFxidWxsZXR7Qn0iXSxbMiw0LCJcXGJ1bGxldHtDfSJdLFs0LDAsIlxcYnVsbGV0e0R9Il0sWzQsMiwiXFxidWxsZXR7RX0iXSxbNCw0LCJcXGJ1bGxldHtGfSJdLFs2LDIsIlxcYnVsbGV0e1R9Il0sWzcsNCwiNyIsMV0sWzcsNSwiNSIsMV0sWzYsNywiMiIsMSx7Im9mZnNldCI6LTJ9XSxbNyw2LCIxIiwxLHsib2Zmc2V0IjotMn1dLFs2LDMsIjMiLDEseyJvZmZzZXQiOjJ9XSxbMywwLCI1IiwxXSxbMiwxLCI2IiwxXSxbMSw1LCIxIiwxXSxbNSw2LCIyIiwxXSxbMSwwLCI0IiwxXSxbMCwyLCI1IiwxLHsib2Zmc2V0IjotMn1dLFs1LDIsIjQiLDFdLFs0LDUsIjMiLDFdLFs0LDEsIjQiLDFdLFs1LDMsIjIiLDFdLFsyLDAsIjQiLDEseyJvZmZzZXQiOi0yfV1d
%\[\begin{tikzcd}[ampersand replacement=\&]
%	\&\& {\bullet{A}} \&\& {\bullet{D}} \\
%	\\
%	{\bullet{S}} \&\& {\bullet{B}} \&\& {\bullet{E}} \&\& {\bullet{T}} \\
%	\\
%	\&\& {\bullet{C}} \&\& {\bullet{F}}
%	\arrow["7"{description}, from=3-7, to=1-5]
%	\arrow["5"{description}, from=3-7, to=3-5]
%	\arrow["2"{description}, shift left=2, from=5-5, to=3-7]
%	\arrow["1"{description}, shift left=2, from=3-7, to=5-5]
%	\arrow["3"{description}, shift right=2, from=5-5, to=5-3]
%	\arrow["5"{description}, from=5-3, to=3-1]
%	\arrow["6"{description}, from=3-3, to=1-3]
%	\arrow["1"{description}, from=1-3, to=3-5]
%	\arrow["2"{description}, from=3-5, to=5-5]
%	\arrow["4"{description}, from=1-3, to=3-1]
%	\arrow["5"{description}, shift left=2, from=3-1, to=3-3]
%	\arrow["4"{description}, from=3-5, to=3-3]
%	\arrow["3"{description}, from=1-5, to=3-5]
%	\arrow["4"{description}, from=1-5, to=1-3]
%	\arrow["2"{description}, from=3-5, to=5-3]
%	\arrow["4"{description}, shift left=2, from=3-3, to=3-1]
%\end{tikzcd}\]
	
	What is the current flow from $S$ to $T$? 
    \Freeform{13}

    Is the current flow the max-flow? 
    \Choices {
        \Choice{Yes}
        \Choice*{No}
    }

    What is the value of max-flow? 
    \Freeform{14}

    What is the only remaining augmenting path in this residual graph? Write it as a string of vertices in the path. For instance if the path is from $S$ to $A$ to $D$ to $T$ you have to write ``SADT'' without the quotation marks.
    \Freeform{SBAEFT}
    
	\section{Matching}
    The company Algorithmia is preparing to transition back to the office in-person work.
    
    They have $n$ offices and $m$ employees that they need to assign to these offices. Given COVID restrictions, each office $i$ has a capacity $C_i$.
    
    Each employee has a list of offices that are within their driving distance. Algorithmia wants to assign as many employees to offices as possible, as long as, every employee is assigned to a single office which is within the employee's driving distance, and the number of employees assigned to every office is less than or equal to the office's capacity.
    
    You are asked to find the maximum possible number of employees that can be assigned to the offices. 
    
    Can this problem be modeled as max-flow?
    \Choices{
        \Choice*{Yes}
        \Choice{No}
    }

    We want to create a directed graph with two dummy vertices $S$ and $T$, and a set of additional vertices: one vertex for each employee and one vertex for each office. In total there are $n + m + 2$ vertices. We want the max-flow from $S$ to $T$ in this graph to be equal to the maximum number of employees that can be assigned to the offices. Please answer the following questions to construct the edges in this graph.
    
    For each employee $i$, we will add an edge from $S$ to $i$. What should be the capacity of this edge? 
    \Choices {
        \Choice{$+\infty$}
        \Choice{$m$}
        \Choice{$n$}
        \Choice*{$1$}
        \Choice{the number of offices that are within driving distance of the employee.}
        \Choice{All of the above are valid.}
    }
    
    For each employee $i$ and office $j$ that is within $i$'s driving distance, we will add an edge from $i$ to $j$. What should be the capacity of this edge? 
    \Choices {
        \Choice{$+\infty$}
        \Choice{$m$}
        \Choice{$n$}
        \Choice{$1$}
        \Choice{The office $j$'s capacity $C_j$.}
        \Choice*{All of the above are valid.}
    }

    For each office $j$ we will add an edge from $j$ to $T$. What should be the capacity of this edge? 
    \Choices {
        \Choice{$+\infty$}
        \Choice{$m$}
        \Choice{$n$}
        \Choice{$1$}
        \Choice*{The office $j$'s capacity $C_j$.}
        \Choice{All of the above are valid.}
    }
    
    \section{Distinct Flows}
    Suppose that we have a graph on $n$ nodes with capacities on the edges. For each pair of nodes $u$ and $v$ we compute the maximum flow with $u$ as a source and $v$ as a sink. In total we collect $n(n-1)$ numbers. If $x$ is the minimum among these $n(n-1)$ numbers, how many times is $x$ repeated in this list? Choose the smallest correct range:
    \Choices{
    	\Choice{$\geq 2$ times.}
    	\Choice{$\geq 1$ times.}
    	\Choice{$= n(n-1)$ times.}
    	\Choice*{$\geq n-1$ times.}
    	\Choice{$\geq n$ times.}
    }
}