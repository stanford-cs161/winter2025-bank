\documentclass{bamboo}

\usepackage{nicematrix}
\Document{
	\Title{Dynamic Programming II}
    
	\section{Longest Common Subsequence Forensics}
	We are computing the longest common subsequence between two strings of length four $S=X_1X_2X_3X_4$ and $T=Y_1Y_2Y_3Y_4$. We fill the array $C$ where $C_{i,j}$ is the length of the longest common subsequence between the prefix of length $i$ from $S$ and the prefix of length $j$ from $T$. The array $C$ can be found below with some entries masked:
	\[
		\begin{bNiceMatrix}[first-row,first-col]
			  & 0 & 1 & 2 & 3 & 4\\
			0 & 0 & 0 & 0 & 0 & 0\\
			1 & 0 & 0 & 0 & 0 & 1\\
			2 & 0 & 0 & 0 & 1 & 1\\
			3 & 0 & 0 & 1 & 1 & \heartsuit\\
			4 & 0 & 1 & \clubsuit & \spadesuit & \diamondsuit \\
		\end{bNiceMatrix}
	\]
	What can be said about $X_1$ and $Y_4$?
	\Choices{
		\Choice*{They are equal.}
		\Choice{They are different.}
		\Choice{They could be equal or different.}
	}
	What can be said about $X_2$ and $Y_3$?
	\Choices{
		\Choice*{They are equal.}
		\Choice{They are different.}
		\Choice{They could be equal or different.}
	}
	What can be said about $X_2$ and $Y_4$?
	\Choices{
		\Choice{They are equal.}
		\Choice*{They are different.}
		\Choice{They could be equal or different.}
	}
	What is the value of $\heartsuit$?
	\Choices{
		\Choice{0}
		\Choice*{1}
		\Choice{2}
		\Choice{3}
		\Choice{4}
		\Choice{Multiple answers could be correct.}
	}
	What is the value of $\spadesuit$?
	\Choices{
		\Choice{0}
		\Choice*{1}
		\Choice{2}
		\Choice{3}
		\Choice{4}
		\Choice{Multiple answers could be correct.}
	}
	What is the value of $\clubsuit$?
	\Choices{
		\Choice{0}
		\Choice*{1}
		\Choice{2}
		\Choice{3}
		\Choice{4}
		\Choice{Multiple answers could be correct.}
	}
	What is the value of $\diamondsuit$?
	\Choices{
		\Choice{0}
		\Choice*{1}
		\Choice{2}
		\Choice{3}
		\Choice{4}
		\Choice{Multiple answers could be correct.}
	}
	Suppose that in some (possibly different) instance of the longest common subsequence problem, we have $C_{i,j}=C_{i-1,j-1}+1$. Does that necessarily mean the $i$-th character of the first string and the $j$-th character of the second string are equal?
	\Choices{
		\Choice{Yes}
		\Choice*{No}
	}
	\section{LCS Space Complexity}
    Consider the LCS problem from lecture $13$ and our dynamic programming algorithm for it. 
    Given input stings of lengths $m$ and $n$, what is the memory complexity of this algorithm?
    \Choices{
		\Choice{$O(n + m)$}
		\Choice{$O((n + m)^2)$}
		\Choice*{$O(mn)$}
	}

    When we are filling up the $i$-th row of our dynamic programming table $C$, what rows do we need to have access to? 
    \Choices{
        \Choice{We need to access all the $n$ rows.}
        \Choice{We need to access the first $i$ rows.}
		\Choice*{We need to access the values in the $i$-th row and $(i-1)$-th row.}
	}

    Given the observation above can we optimize our space Complexity further?
    \Choices{
        \Choice{No, the best memory Complexity is $O(nm)$}
        \Choice{Yes we can reduce the memory complexity to $O(n\log(m))$.}
		\Choice{Yes we can reduce the memory complexity to $O(\frac{m}{n})$.}
		\Choice*{Yes we can reduce the memory complexity to $O(\min(m, n))$.}
	}
	\section{Knapsack Forensics}
	Suppose we are trying to solve an instance of the unbounded knapsack problem. We fill the array $K$ whose entry $K_i$ gives us the maximum value we can obtain from a knapsack of capacity $i$.
	\[ \begin{bNiceMatrix}[first-row]
		0 & 1 & 2 & 3 & 4 & 5 & 6\\
		0 & 0 & 1 & 3 & \spadesuit & \heartsuit & \diamondsuit \\
 	\end{bNiceMatrix}\]
 	What is the minimum possible value for $\spadesuit$?
 	\Freeform{3}
 	What is the minimum possible value for $\heartsuit$?
 	\Freeform{4}
 	What is the minimum possible value for $\diamondsuit$?
 	\Freeform{6}
 	If we fill a knapsack of capacity $3$ optimally, how many items do we put in the knapsack?
 	\Choices{
 		\Choice{0}
 		\Choice*{1}
 		\Choice{2}
 		\Choice{3}
 		\Choice{Multiple answers could be correct.}
 	}
 	
 	\section{Maximum Independent Set on a Tree} 
    Consider the maximum independent on trees problem from the lecture $13$ slides. 
    We saw a top-down dynamic programming approach to solve this problem.
    Now we'd like to see how a bottom up approach to solve MIS on a tree would look like.  
    Which one of the following statements is correct? 

    \Choices{
		\Choice{In order to solve this problem bottom-up we need to order the vertices by increasing DFS finish time.}
		\Choice{In order to solve this problem bottom-up we need to order the vertices by decreasing DFS start time.}
		\Choice*{Both of the above.}
        \Choice {Any ordering would work.}

	}

    What is the best run-time for the bottom-up approach to solve the MIS on a tree problem? 
    \Choices{
		\Choice*{$O(n + m)$}
		\Choice{$O((n + m)\log(n))$}
		\Choice{$O((n + m)^2)$}
	}
}
